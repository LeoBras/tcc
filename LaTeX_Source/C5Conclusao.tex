\chapter{Conclus�o}

Diante dos resultados obtidos no Cap�tulo ~\ref{ch:res}, pode-se observar que os objetivos propostos no in�cio do trabalho foram atingidos, obtendo uma plataforma de baixo custo, tempo de resposta com previsibilidade adequada e alta flexibilidade para a utiliza��o em fins de automa��o.

Por meio de resultados obtidos nos testes, pode-se observar que a plataforma desenvolvida atenderia a sistemas de Tempo-Real que exijam menos de 500us de tempo de resposta, pois o tempo de lat�ncia m�ximo encontrado ap�s quase 48h de teste foi menor que 250us. Esses valores s�o atrativos, levando-se em considera��o que o hardware � bastante limitado.

Uma sugest�o para trabalhos futuros � estudar o comportamento da plataforma em rela��o � necessidade de aplica��es que utilizem Multi-Tarefas em Real-Time.

Outra possibilidade de estudo futuro � a an�lise da qualidade das tarefas que n�o s�o executadas em Real-Time, e como a poss�vel perda de desempenho, em fun��o do atendimento ao Real-Time, pode afetar a qualidade desses servi�os.

H� boas estimativas de que, com componentes mais atuais e mais tempo de desenvolvimento do \textit{patch} RT, a plataforma atinja boa conceitua��o e passe a ser considerada refer�ncia para automa��o em tempo real.


	
	
